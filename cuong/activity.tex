%! TEX root = ../main.tex
\documentclass[../main.tex]{subfiles}

\begin{document}

\subsection{Ca sử dụng: Tạo bài viết}

\begin{figure}[H]{Biểu đồ hoạt động Tạo bài viết}
    \centering
    \includegraphics[width=0.75\textwidth]{acd_upload post.png}
    \label{fig:act1}
\end{figure}

Biểu đồ \ref{fig:act1} mô tả các bước hoạt động của ca sử dụng “Tạo bài viết”. Người dùng thông qua việc đăng bài có thể tương tác với diễn đàn, chia sẻ kiến thức hoặc đặt câu hỏi cho người khác. Ở giao diện màn hình chính, người dùng truy cập nhanh vào tính năng này bằng nút “Tạo bài viết” ở chính giữa thanh điều hướng. Một thẻ giao diện tạo bài sẽ hiện lên, hướng dẫn người dùng tạo bài qua 3 bước. Đầu tiên, người dùng chọn một trong số các chủ đề mà hệ thống cung cấp. Tiếp theo, người dùng nhập tiêu đề và nội dung bài viết theo biểu mẫu. Sau đó, người dùng có thể chọn tối đa 5 ảnh cho bài viết của mình, hoặc không thêm ảnh và tiếp tục. Cuối cùng, giao diện sẽ hiển thị bài viết dưới dạng xem trước để người dùng kiểm tra lần cuối trước khi gửi. Người dùng xác nhận gửi, bài viết sẽ được chuyển đến máy chủ để xử lí và lưu trữ. Khi bài viết được tải lên thành công, giao diện sẽ chuyển qua xem bài viết mà người dùng vừa đăng.

\subsection{Ca sử dụng: Xem bài viết}

\begin{figure}[H]{Biểu đồ hoạt động Xem bài viết}
    \centering
    \includegraphics[width=0.75\textwidth]{acd_view post.png}
    \label{fig:act2}
\end{figure}

Người dùng có thể xem các bài viết để cùng thảo luận, học tập và chia sẻ kiến thức. Ở các màn hình trang chủ, tìm kiếm, hay xem hồ sơ của một người, người dùng có thể xem trước các bài viết trên diễn đàn. Để xem đầy đủ nội dung bài viết cũng như các bình luận, người dùng chọn vào bài viết đó để mở giao diện xem chi tiết. Khi đó, hệ thống sẽ kiểm tra xem bài viết có còn tồn tại hay đã bị xóa. Nếu đã bị xóa, hệ thống thông báo không thể truy cập bài viết này. Còn nếu bài viết vẫn tồn tại, hệ thống sẽ tải thông tin bài viết, cũng như các bình luận đến người dùng. Sau khi tải xong, ứng dụng sẽ thiết lập kết nối liên tục đến máy chủ để nhận các cập nhật mới (bình luận mới, bình chọn mới) cho bài viết đang xem trong thời gian thực. Trong giao diện xem bài viết này, người dùng có thể cuộn xuống để đọc bình luận, thực hiện các ca sử dụng bình chọn, đăng tải bình luận của mình, phản hồi bình luận người khác, hoặc tạo câu hỏi liên quan đến nội dung bài viết bằng AI.

\subsection{Ca sử dụng: Tạo bình luận}

\begin{figure}[H]{Biểu đồ hoạt động Tạo bình luận}
    \centering
    \includegraphics[width=0.95\textwidth]{acd_cmt.png}
    \label{fig:act3}
\end{figure}

Sau khi mở xem chi tiết bài viết, người dùng có thể đưa ra ý kiến của mình thông qua bình luận. Người dùng được cung cấp một ô bình luận ở cạnh dưới màn hình. Ngoài ra họ có thể nhấn “Phản hồi” trên một bình luận để chuyển sang chế độ phản hồi. Sau khi nhập nội dung bình luận, người dùng nhấn vào biểu tượng máy bay giấy ở bên cạnh để gửi bình luận. Nếu chẳng may bài viết đã bị xóa, hoặc bình luận đang được phản hồi bị xóa, hệ thống sẽ thông báo rằng nội dung không thể truy cập được. Nếu không, bình luận sẽ được hệ thống lưu lại và hiển thị trên giao diện của người dùng, và cập nhật trên giao diện của các người dùng khác nếu họ cũng đang xem bài viết này thông qua luồng phản hồi SSE. Ngoài ra, với bình luận thông thường, chủ bài viết sẽ nhận được thông báo có bình luận mới, còn với phản hồi thì chủ bình luận sẽ nhận được thông báo có phản hồi mới.

\end{document}