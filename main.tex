% !TEX program = lualatex
% !TEX options = -lualatex

\documentclass{uetgraduation}
\usepackage{enumitem}
% Bảng kiểm thử: màu nền header, tabularx, đường kẻ dày/mảnh, ô đa dòng
\usepackage{tabularx}
% Cho bảng dài tự ngắt trang: kết hợp tabularx + longtable
\usepackage{ltablex}
\keepXColumns
\usepackage{array}
\usepackage{makecell}
\usepackage{multirow}
\usepackage[hidelinks]{hyperref} 
\usepackage{xurl}
\urlstyle{same} 
\usepackage{unicode-math}
\setmathfont{Cambria Math}
\begin{document}

\studentname{23020122 -- Phùng Hải Nam}
\studentnametwo{23020122 -- Nguyễn Văn Cường}
\studentnamethree{23020122 -- Tô Quang Thắng}
\title{Ứng dụng EnglishForum}
\documenttype{Báo cáo Phát triển ứng dụng di động}
\major{Công nghệ thông tin}
\year{2025}
\supervisor{PGS. TS. Nguyễn Đông A}
\cosupervisor{TS. Phạm Tuấn B}

\makecovers

\begin{contentlisting}
    \begin{contentlistingsection}{}
        \listoftables
    \end{contentlistingsection}

    \begin{contentlistingsection}{}
        \listoffigures
    \end{contentlistingsection}

    \begin{contentlistingsection}{Danh sách thuật ngữ}
        % Thêm các mục thuật ngữ dưới đây theo định dạng \item[Thuật ngữ] Mô tả
        \begin{description}
            \item[hehe] so hehe
        \end{description}
    \end{contentlistingsection}

    \begin{contentlistingsection}{Lời cam đoan}
        {\indent
        Chúng tôi xin cam đoan rằng nội dung khóa luận này là kết quả lao động cá nhân của cả nhóm,
        trừ những phần đã được trích dẫn và tham khảo rõ ràng. Chúng chịu trách nhiệm về
        những nội dung thuộc về mình và xin cam kết không vi phạm các quy định về đạo đức
        học thuật. Nếu phát hiện 1 đoạn do AI sinh ra, cả nhóm xin nhận 0 điểm.\par}

        \vspace{2cm}
        \begin{center}
            \begin{minipage}[t]{0.3\textwidth}
                \centering
                \textbf{Chữ ký học sinh}
            \end{minipage}\hfill
            \begin{minipage}[t]{0.3\textwidth}
                \centering
                \textbf{Chữ ký học sinh}
            \end{minipage}\hfill
            \begin{minipage}[t]{0.3\textwidth}
                \centering
                \textbf{Chữ ký học sinh}
            \end{minipage}
        \end{center}
    \end{contentlistingsection}
    
    % Mục lục ngay sau Lời cam đoan
    \tableofcontents
    
\end{contentlisting}


% Bắt đầu nội dung chính (chương 1)
\chapter{Đặt vấn đề}\label{ch:dat-van-de}


\noindent Đây là phần mở đầu chương 1 (Đặt vấn đề). Thêm nội dung chi tiết về bối cảnh,
vấn đề nghiên cứu, mục tiêu và phạm vi ở đây.
%--------------------------------
\section{Hiện trạng}
Tiếng anh từ lâu đã dần trở thành 1 ngôn
 ngữ quan trọng mà mọi học sinh, sinh viên 
 đều đầu tư rất nhiều thời gian để làm chủ.
 Trong giai đoạn 2025 -- 2035, Bộ Giáo duc và
 Đào tạo dự kiến 100\% học sinh được học tiếng anh
 như ngôn ngữ thứ hai. Hiện nay, 
mặc dù đã có nhiều trang trao đổi các vấn
 đề về tiếng anh nhưng đa phần đều hoặc đã rất
  cũ, hoặc không được cập nhật
 thường xuyên, hoặc chỉ là 1 phần nhỏ trong 
 một phần mềm lớn, bị làm lu mờ bởi rất nhiều chủ đề trao đổi khác.\par
 Vấn đề dần lộ ra rõ đối với người học tiếng anh. Họ thiếu môi trường tương tác, 
 khó tìm đối tác học tập, không có nơi để trao đổi các vấn đề về tiếng 
 anh một cách hiệu quả.
  Điều này dẫn đến việc học tiếng anh trở nên nhàm chán, 
  thiếu động lực và không đạt được hiệu quả cao.\par
  Với việc các ứng dụng trên di dộng ngày càng được ưa chuộng do tính
  dễ dùng, tiện lợi của chúng, cộng với việc quan sát được sự thành công của 
  các nền tảng diễn đàn trực tuyến chuyên biệt cho người dân
  nội địa như 2channel\cite{2chan} ở Nhật Bản hay Dcard\cite{Dcard} ở Đài Loan, việc xây 
  dựng một diễn đàn trên di động chuyên
  biệt cho việc trao đổi và học tập tiếng anh được đánh giá là một phương án
   khả thi.
   Nền tảng này sẽ cung cấp môi trường tương tác, kết 
   nối người học với nhau, giúp họ chia sẻ kiến thức,
    kinh nghiệm và hỗ trợ lẫn nhau trong quá trình học tập.
%--------------------------------
\section{Các giải pháp đã có}

\subsection{Mô tả}
Hiện tại có rất nhiều trang web trực tuyến với mục đích hỗ trợ việc học tiếng anh. 
Các trang web này cung cấp các bài học, tài liệu, và công cụ để người học có thể nâng cao kỹ năng tiếng anh của mình.
 Một số trang web nổi bật kể đến như Duolingo, Memrise, \ldots. Các 
 trang web này thường cung cấp các bài học tương tác, trò chơi học tập, và các bài kiểm tra để giúp người học theo dõi tiến trình của mình.\par
 Ngoài ra, trên những nền tảng mạng xã hội cho phép tạo lập hội nhóm cũng hình
 thành nhiều cộng đồng học tiếng anh. Các nhóm này cũng có mục đích tương tự là kết nối người học với nhau để trao đổi kiến thức và kinh nghiệm học tập.


\subsection{Hạn chế}
Các giải pháp hỗ trợ học tiếng anh hiện tại đã phần nào hỗ trợ khá tốt việc tự học
tiếng anh. Tuy nhiên, chưa tạo được một môi trường cùng học tập tiếng anh giữa
các người học với nhau một cách hiệu quả. Nhiều nền tảng chỉ tập trung vào việc
 cung cấp tài liệu học tập mà thiếu đi các tính năng tương tác, trao đổi giữa người học.\par
 Giải pháp sử dụng mạng xã hội làm cầu nối đã tạo được tính tương tác giữa người học với nhau,
 nhưng quy mô còn nhỏ và thường là tự tổ chức giữa các thành viên trong nhóm, làm giảm khả năng tiếp cận
 của người học mới.

%--------------------------------
\section{Báo cáo}
% Tổng hợp kết quả kiểm thử
\subsection{Kết quả kiểm thử hiệu}
Trình bày kết quả ``kiểm thử hiệu năng''

\subsection{Kết quả kiểm thử hệ thống}
Trình bày kết quả kiểm thử hệ thống

%--------------------------------
\chapter{Kiến thức nền tảng}

\chapter{Thu thập phân tích đặc tả yêu cầu}

\section{Biểu đồ ca sử dụng hệ thống EnglishForum}

\section{Mô tả ca sử dụng}
\subsection{Ca sử dụng \textit{Tạo bài viết}}
\subsection{Ca sử dụng \textit{Xem bài viết}}
\subsection{Ca sử dụng \textit{Bình luận bài viết}}
\section{Biểu đồ hoạt động}
\subsection{Ca sử dụng \textit{Tạo bài viết}}
\subsection{Ca sử dụng \textit{Xem bài viết}}
\subsection{Ca sử dụng \textit{Bình luận bài viết}}
\chapter{Thiết kế hệ thống}
\section{Thiết kế mức cao}
\section{Biểu đồ tuần tự}
\section{Test}

% Các bảng kiểm thử tách riêng theo chức năng
\subsection{Đăng ký tài khoản}
% Nội dung bảng nằm ở tables/dang-ky-tai-khoan.tex

\subsection{Đăng nhập tài khoản}
% Nội dung bảng nằm ở tables/dang-nhap-tai-khoan.tex

\subsection{Tạo bài viết}
% Nội dung bảng nằm ở tables/tao-bai-viet.tex
% !TeX root = ../main.tex
% Bảng dài tự ngắt trang: dùng ltablex (tabularx dạng longtable) và không đặt trong float
% In caption theo style của class `uetgraduation`
{\makeatletter
 \refstepcounter{a@table} % tăng số bảng
 \def\@captype{a@table} % đặt loại caption là bảng
 \def\@captiontext{\centering Bảng kiểm thử: Tạo bài viết} % nội dung caption
 \noindent\attachmentcaption\par\vspace{0.25cm} % in caption và giãn cách
\makeatother}

\setlength{\tabcolsep}{5pt}
\renewcommand{\arraystretch}{1.25}
\setlength{\arrayrulewidth}{1pt}

\begin{tabularx}{\textwidth}{|
  >{\centering\arraybackslash}p{1.2cm}|
  >{\raggedright\arraybackslash}X|
  >{\raggedright\arraybackslash}X|
  >{\raggedright\arraybackslash}X|
  >{\raggedright\arraybackslash}X|
  >{\centering\arraybackslash}p{2.2cm}|
}
\hline
\multicolumn{3}{|c|}{} &\multicolumn{3}{|c|}{\parbox[t]{8cm}{\raggedright Ngày thực hiện:\\ Người thực hiện:}}\\
\hline
\multirow{2}{*}{STT} & \multicolumn{1}{|c|}{\multirow{2}{*}{Mô tả}} & \multicolumn{1}{|c|}{\multirow{2}{*}{Đầu vào}} & \multicolumn{2}{c|}{Đầu ra} & \multirow{2}{*}{Trạng thái}\\
\cline{4-5}
 &  &  & Đầu ra mong muốn & Đầu ra thực tế & \\
\hline
\endfirsthead

\hline
\multirow{2}{*}{STT} & \multicolumn{1}{|c|}{\multirow{2}{*}{Mô tả}} & \multicolumn{1}{|c|}{\multirow{2}{*}{Đầu vào}} & \multicolumn{2}{c|}{Đầu ra} & \multirow{2}{*}{Trạng thái}\\
\cline{4-5}
 &  &  & Đầu ra mong muốn & Đầu ra thực tế & \\
\hline
\endhead

1 & Tạo bài viết hợp lệ (đầy đủ thông tin) & Title = ``Learning English'', Body = ``Here are some tips...'', Tag = ``Hỏi đáp'', không có ảnh & Bài viết được tạo thành công, hiển thị snackbar ``Đăng bài thành công'' và điều hướng tới bài viết vừa được tạo & Như mong đợi & Pass \\
\hline
2 & Không nhập tiêu đề & Title = ``'', Body = ``This is content.'', Tag = ``Hỏi đáp'' & Nút ``Tiếp tục'' bị ẩn đi & Như mong đợi & Pass \\
\hline
3 & Không nhập nội dung & Title = ``Learning English'', Body = ``'', Tag = ``Hỏi đáp'', không có ảnh & Nút ``Tiếp tục'' bị ẩn đi & Như mong đợi & Pass \\
\hline
4 & Thêm 1 ảnh hợp lệ & Title = ``Learning English'', Body = ``Here are some tips'', Tag = ``Hỏi đáp'', Ảnh = tùy chọn với kích thước ≤ 5MB & Bài viết được tạo thành công với hình ảnh đính kèm, điều hướng tới bài viết vừa được tạo & Như mong đợi & Pass \\
\hline
5 & Thêm 5 ảnh hợp lệ & Title = ``Learning English'', Body = ``Here are some tips'', Tag = ``Hỏi đáp'', Ảnh = 5 ảnh tùy chọn với kích thước ≤ 5MB & Bài viết được tạo thành công với hình ảnh đính kèm, điều hướng tới bài viết vừa được tạo & Như mong đợi & Pass \\
\hline
6 & Vượt giới hạn số ảnh & Title = ``Learning English'', Body = ``Here are some tips'', Tag = ``Hỏi đáp'', Ảnh = 6 ảnh tùy chọn với kích thước ≤ 5MB & Nút ``Thêm ảnh'' bị ẩn đi khi thêm tới ảnh thứ 5 & Như mong đợi & Pass \\
\hline
7 & Ảnh sai định dạng & Title = ``Learning English'', Body = ``Here are some tips'', Tag = ``Hỏi đáp'', Ảnh = sample với đuôi \text{.heic}/\allowbreak\text{.tiff}/\allowbreak\text{.webp} & Có thông báo ``Chỉ hỗ trợ JPG, PNG, GIF'' và không cho phép tải lên & Như mong đợi & Pass \\
\hline
8 & Ảnh quá dung lượng & Title = ``Learning English'', Body = ``Here are some tips'', Tag = ``Hỏi đáp'', Ảnh có kích thước 6 MB & Có thông báo ``Kích thước ảnh phải ≤ 5MB'' và không cho phép tải lên & Như mong đợi & Pass \\
\hline
9 & Xóa 1 ảnh khỏi danh sách & Title = ``Learning English'', Body = ``Here are some tips'', Tag = ``Hỏi đáp'', Ảnh = 1 ảnh tùy chọn với kích thước ≤ 5MB & Ảnh được thêm bị xóa khỏi danh sách ảnh & Như mong đợi & Pass \\
\hline
10 & Đăng bài khi không có kết nối mạng & Title = ``Offline test'', Body = ``Should fail'', Tag = ``Hỏi đáp'', Ảnh = 1 ảnh tùy chọn với kích thước ≤ 5MB. Tắt kết nối mạng. & Không thể kết nối tới máy chủ. Vui lòng kiểm tra lại mạng & Như mong đợi & Pass \\
\hline
11 & Tiếp tục đăng bài khi có kết nối mạng trở lại sau khi mất kết nối & Title = ``Offline test'', Body = ``Should fail'', Tag = ``Hỏi đáp'', Ảnh = 1 ảnh tùy chọn với kích thước ≤ 5MB. Tắt kết nối mạng, nhấn ``Đăng bài'' thất bại và bật lại kết nối mạng. & Sau khi bật lại mạng lên và nhấn đăng bài, bài viết được đăng thành công, hiển thị snackbar ``Đăng bài thành công'' và điều hướng tới bài viết vừa được tạo & Như mong đợi & Pass \\
\hline
12 & Làm mới biểu mẫu tạo bài viết sau khi đăng bài thành công & Title = ``Learning English'', Body = ``Here are some tips'', Tag = ``Hỏi đáp'', Ảnh = 1 ảnh tùy chọn với kích thước ≤ 5MB & Sau khi đăng bài thành công, biểu mẫu tạo bài viết được làm mới & Như mong đợi & Pass \\
\hline
13 & Biểu mẫu tạo bài viết được lưu trạng thái khi chưa đăng bài viết & Title = ``Learning English'', Body = ``Here are some tips'', Tag = ``Hỏi đáp'', Ảnh = 1 ảnh tùy chọn với kích thước ≤ 5MB & Tắt bỏ và mở lại biểu mẫu tạo bài viết, trạng thái được giữ lại & Như mong đợi & Pass \\
\hline
14 & Làm mới các bài viết & Tạo một bài viết mới, ra trang chủ thực hiện thao tác kéo xuống & Tất cả các bài viết được làm mới, cập nhật thông tin và thêm bài viết mới & Như mong đợi & Pass \\
\hline


\end{tabularx}


\subsection{Xem bài viết}\
% Nội dung bảng nằm ở tables/xem-bai-viet.tex
% !TeX root = ../main.tex
% Bảng dài tự ngắt trang: dùng ltablex (tabularx dạng longtable) và không đặt trong float
% In caption theo style của class `uetgraduation`
{\makeatletter
 \refstepcounter{a@table} % tăng số bảng
 \def\@captype{a@table} % đặt loại caption là bảng
 \def\@captiontext{\centering Bảng kiểm thử: Xem bài viết} % nội dung caption
 \noindent\attachmentcaption\par\vspace{0.25cm} % in caption và giãn cách
\makeatother}

\setlength{\tabcolsep}{5pt}
\renewcommand{\arraystretch}{1.25}
\setlength{\arrayrulewidth}{1pt}

\begin{tabularx}{\textwidth}{|
  >{\centering\arraybackslash}p{1.2cm}|
  >{\raggedright\arraybackslash}X|
  >{\raggedright\arraybackslash}X|
  >{\raggedright\arraybackslash}X|
  >{\raggedright\arraybackslash}X|
  >{\centering\arraybackslash}p{2.2cm}|
}
\hline
\multicolumn{3}{|c|}{} &\multicolumn{3}{|c|}{\makecell[l]{Execute Date:\\ Tester:}}\\
\hline

\multirow{2}{*}{ID} & \multicolumn{1}{|c|}{\multirow{2}{*}{Description}} & \multicolumn{1}{|c|}{\multirow{2}{*}{Input}} & \multicolumn{2}{c|}{Output} & \multirow{2}{*}{Test status}\\
\cline{4-5}
 &  &  & Expected Output & Actual Output & \\
\hline

1 & Xem một bài viết hợp lệ có 1 ảnh & Bài viết có title, body, tag, có 1 ảnh &  Điều hướng đến màn hình bài viết chi tiết, bao gồm ảnh đại diện và tên tác giả, thời gian đăng tương đối, tag, title, body, ảnh, bộ đếm vote và bình luận & Như mong đợi & Pass \\
\hline
2 & Hiển thị ảnh xem trước & Bài viết có title, body, tag, có 1 ảnh & Hiển thị ảnh xem trước ở ngoài bài viết hợp lệ & Như mong đợi & Pass \\
\hline
3 & Xem bài viết hợp lệ có nhiều ảnh & Bài viết có title, body, tag, có nhiều hơn 1 ảnh & Điều hướng đến màn hình bài viết chi tiết, bao gồm ảnh đại diện và tên tác giả, thời gian đăng tương đối, tag, title, body, thư viện ảnh có thể vuốt trái phải để xem ảnh, bộ đếm vote và bình luận & Như mong đợi & Pass \\
\hline
4 & Phóng to ảnh toàn màn hình & Bài viết có title, body, tag, có 1 ảnh, thao tác pinch and pan & Ảnh phóng to tối đa, giữ đúng tỉ lệ. Nút đóng ảnh hoạt động. Thao tác pinch and pan hoạt động. & Như mong đợi & Pass \\
\hline
5 & Kéo xuống để làm mới & Bài viết có title, body, tag, có 1 ảnh. Chủ bài viết thay đổi title trước khi tester thực hiện thao tác kéo xuống & Hiển thị biểu tượng làm mới, nội dụng bài viết được cập nhật đúng với thay đổi & Như mong đợi & Pass \\
\hline
6 & Xem bài viết đã bị xóa & Bài viết có title, body, tag, có 1 ảnh. Chủ bài viết xóa bài viết trong khi tester đang xem chi tiết bài viết và thực hiện thao tác kéo xuống làm mới & Hiển thị snackbar "Post not found" & Như mong đợi & Pass \\
\hline %can test them
7 & Xem một bài viết hợp lệ khi không có kết nối mạng & Bài viết có title, body, tag, có 1 ảnh. Tắt kết nối mạng. & Hiển thị biểu tượng tải bài viết ở giữa màn hình. & Như mong đợi & Pass \\
\hline


\end{tabularx}


\subsection{Sửa bài viết}
% Nội dung bảng nằm ở tables/sua-bai-viet.tex
% !TeX root = ../main.tex
% Bảng dài tự ngắt trang: dùng ltablex (tabularx dạng longtable) và không đặt trong float
% In caption theo style của class `uetgraduation`
{\makeatletter
 \refstepcounter{a@table} % tăng số bảng
 \def\@captype{a@table} % đặt loại caption là bảng
 \def\@captiontext{\centering Bảng kiểm thử: Sửa bài viết} % nội dung caption
 \noindent\attachmentcaption\par\vspace{0.25cm} % in caption và giãn cách
\makeatother}

\setlength{\tabcolsep}{5pt}
\renewcommand{\arraystretch}{1.25}
\setlength{\arrayrulewidth}{1pt}

\begin{tabularx}{\textwidth}{|
  >{\centering\arraybackslash}p{1.2cm}|
  >{\raggedright\arraybackslash}X|
  >{\raggedright\arraybackslash}X|
  >{\raggedright\arraybackslash}X|
  >{\raggedright\arraybackslash}X|
  >{\centering\arraybackslash}p{2.2cm}|
}
\hline
\multicolumn{3}{|c|}{} &\multicolumn{3}{|c|}{\makecell[l]{Execute Date:\\ Tester:}}\\
\hline

\multirow{2}{*}{ID} & \multicolumn{1}{|c|}{\multirow{2}{*}{Description}} & \multicolumn{1}{|c|}{\multirow{2}{*}{Input}} & \multicolumn{2}{c|}{Output} & \multirow{2}{*}{Test status}\\
\cline{4-5}
 &  &  & Expected Output & Actual Output & \\
\hline

1 & Sửa hợp lệ thêm ảnh & Title = ``Learning English – v2”, Body = ``Here are some tips - v2'', Tag đổi thành ``Hướng dẫn'', thêm 1 ảnh mới hợp lệ & Chỉnh sửa thành công, hiển thị snackbar ``Đã cập nhật bài viết'' và điều hướng về màn hình chi tiết bài viết đã được cập nhật đúng nội dung mới & Như mong đợi & Pass \\
\hline
2 & Sửa hợp lệ nhưng mất kết nối & Title = ``Learning English – v3”, Body = ``Here are some tips - v3'', tag và ảnh giữ nguyên & Hiển thị snackbar báo lỗi ``Không thể kết nối tới máy chủ. Vui lòng kiểm tra lại mạng''. & Như mong đợi & Pass \\
\hline
3 & Không thay đổi gì & Giữ nguyên title, body, tag, ảnh & Nhấn ``Lưu thay đổi'' vẫn hiển thị snackbar ``Đã cập nhật bài viết'' và điều hướng về màn hình chi tiết bài viết & Như mong đợi & Pass \\
\hline
4 & Sửa nhưng bỏ trống title & Giữ nguyên body, tag, ảnh nhưng bỏ trống title & Nút lưu thay đổi bị vô hiệu hóa & Như mong đợi & Pass \\
\hline
5 & Sửa nhưng bỏ trống body & Giữ nguyên title, tag, ảnh nhưng bỏ trống body & Nút lưu thay đổi bị vô hiệu hóa & Như mong đợi & Pass \\
\hline
6 & Sửa hợp lệ xóa ảnh & Giữ nguyên title, body, tag nhưng xóa ảnh & Hiển thị snackbar ''Đã cập nhật bài viết'' và điều hướng về màn hình chi tiết bài viết đã được cập nhật đúng nội dung mới (mất ảnh) & Như mong đợi & Pass \\
\hline
7 & Hủy chỉnh sửa & Title = ``Learning English – v4”, Body = ``Here are some tips - v4'', Tag = ``Hướng dẫn'', thêm 1 ảnh mới hợp lệ. Nhấn nút quay trở lại thay vì ``Lưu thay đổi'' & Bài viết không bị chỉnh sửa. Điều hướng về màn hình trước đó & Như mong đợi & Pass \\
\hline
8 & Sửa hợp lệ nhưng title chỉ nhập khoảng trắng & Title = ``  '', Body =``Here are some tips - v4''”, Tag = ``Hướng dẫn'', ảnh giữ nguyên & Nút ``Lưu thay đổi'' bị vô hiệu hóa & Như mong đợi & Pass \\
\hline
9 & Sửa hợp lệ nhưng body chỉ nhập khoảng trắng & Title = ``Learning English – v4'', Body = ``  '', Tag = ``Hướng dẫn'', ảnh giữ nguyên & Nút ``Lưu thay đổi'' bị vô hiệu hóa & Như mong đợi & Pass \\
\hline

10 & Thêm 1 ảnh hợp lệ & Giữ nguyên title, body, tag; thêm 1 ảnh mới hợp lệ khi bài viết đang có dưới 5 ảnh & Hiển thị snackbar ``Đã cập nhật bài viết''; danh sách ảnh gồm ảnh cũ và ảnh mới theo thứ tự thêm & Như mong đợi & Pass \\
\hline
11 & Vượt giới hạn số ảnh & Bài viết đang có 5 ảnh; tiếp tục thêm ảnh thứ 6 & Nút ``Thêm ảnh'' bị vô hiệu hoá & Như mong đợi & Pass \\
\hline
12 & Ảnh sai định dạng & Thêm ảnh đuôi \text{.heic}/\allowbreak\text{.tiff}/\allowbreak\text{.webp} & Hiện cảnh báo ``Chỉ hỗ trợ JPG, PNG, GIF'' và không thêm ảnh & Như mong đợi & Pass \\
\hline
13 & Ảnh quá dung lượng & Thêm ảnh hợp lệ nhưng kích thước >5MB & Hiện cảnh báo ``Kích thước ảnh phải \le 5MB'' và không thêm ảnh & Như mong đợi & Pass \\
\hline

14 & Bài viết đã bị xoá trước khi lưu & Tester thực hiện sửa bài viết nhưng từ thiết bị khác xoá bài, quay lại thiết bị 1 bấm "Lưu thay đổi" & Hiển thị snackbar "Không tìm thấy bài viết" & Như mong đợi & Pass \\
\hline

\end{tabularx}


\subsection{Xóa bài viết}
% Nội dung bảng nằm ở tables/xoa-bai-viet.tex
% !TeX root = ../main.tex
% Bảng dài tự ngắt trang: dùng ltablex (tabularx dạng longtable) và không đặt trong float
% In caption theo style của class `uetgraduation`
{\makeatletter
 \refstepcounter{a@table} % tăng số bảng
 \def\@captype{a@table} % đặt loại caption là bảng
 \def\@captiontext{\centering Bảng kiểm thử: Xóa bài viết} % nội dung caption
 \noindent\attachmentcaption\par\vspace{0.25cm} % in caption và giãn cách
\makeatother}

\setlength{\tabcolsep}{5pt}
\renewcommand{\arraystretch}{1.25}
\setlength{\arrayrulewidth}{1pt}

\begin{tabularx}{\textwidth}{|
  >{\centering\arraybackslash}p{1.2cm}|
  >{\raggedright\arraybackslash}X|
  >{\raggedright\arraybackslash}X|
  >{\raggedright\arraybackslash}X|
  >{\raggedright\arraybackslash}X|
  >{\centering\arraybackslash}p{2.2cm}|
}
\hline
\multicolumn{3}{|c|}{} &\multicolumn{3}{|c|}{\makecell[l]{Execute Date:\\ Tester:}}\\
\hline

\multirow{2}{*}{ID} & \multicolumn{1}{|c|}{\multirow{2}{*}{Description}} & \multicolumn{1}{|c|}{\multirow{2}{*}{Input}} & \multicolumn{2}{c|}{Output} & \multirow{2}{*}{Test status}\\
\cline{4-5}
 &  &  & Expected Output & Actual Output & \\
\hline

1 & Xóa bài viết hợp lệ & Tạo một bài viết thực hiện xóa bài viết bằng tùy chọn trên giao diện xem chi tiết bài viết & Có thông báo xác nhận xóa và bài viết được xóa không hiển thị trên diễn đàn sau khi xác nhận & Như mong đợi & Pass \\
\hline
2 & Hủy Xóa bài viết & Tạo một bài viết thực hiện xóa bài viết bằng tùy chọn trên giao diện xem chi tiết bài viết nhưng hủy xóa khi có thông báo xác nhận & Có thông báo xác nhận xóa, bài viết vẫn hiển thị trên diễn đàn sau khi nhấn hủy & Như mong đợi & Pass \\
\hline

\end{tabularx}


% Tài liệu tham khảo 
%Các tài liệu tham khảo khi liệt kê vào danh mục phải đầy đủ 
%các thông tin cần thiết và theo trình tự sau: 
%Số thứ tự (đặt trong cặp dấu ngoặc vuông), 
%Họ tên tác giả,
% Tên tài liệu (bài báo, sách, ...),
% Nguồn (tên tạp chí, tập, số, năm hoặc tên nhà xuất bản), 
%Trang tham khảo
\begin{thebibliography}{99}
    \begin{bibsection}{Tiếng Việt}
        \bibitem{hehe}
            hehe, \textit{The Art of hehe}, .

    \end{bibsection}

    \begin{bibsection}{Tiếng Anh}
        \bibitem{2chan}
            ``2channel'', \textit{Wikipedia}, Wikipedia Foundation, 4 October 2025, \url{https://en.wikipedia.org/wiki/2channel}
        \bibitem{Dcard}
            ``Dcard'', \textit{Wikipedia}, Wikipedia Foundation, 29 October 2025, \url{https://en.wikipedia.org/wiki/Dcard}
    \end{bibsection}
\end{thebibliography}

\end{document}

