% !TEX program = lualatex
% !TEX options = -lualatex

\documentclass{uetgraduation}
\usepackage{enumitem}

\begin{document}

\studentname{Nhóm chuốiiii}
\title{Báo cáo Phát triển ứng dụng di động}
\documenttype{Khóa luận tốt nghiệp đại học hệ chính quy}
\major{Công nghệ thông tin}
\year{2024}
\supervisor{PGS. TS. Nguyễn Đông A}
\cosupervisor{TS. Phạm Tuấn B}

\makecovers


\begin{preamble}{Tóm tắt}
    \textbf{Tóm tắt:} Hiện nay, với sự phát triển nhanh chóng của các dịch vụ IP và
    sự bùng nổ của Internet đã dẫn đến một loạt các vấn đề được đặt ra như: tốc độ
    truyền, quản lý chất lượng dịch vụ, điều phối dung lượng... Gần đây, công nghệ
    chuyển mạch nhãn đa giao thức MPLS được đề xuất, MPLS đã kết hợp được khả năng
    định tuyến tốt ở lớp 3 và chuyển mạch ở lớp 2, nó mở ra một viễn cảnh cho rất
    nhiều ứng dụng quan trọng. Mạng riêng ảo VPN là một trong những ứng dụng nổi
    bật nhất của công nghệ MPLS, MPLS VPN đã khắc phục được hầu hết những nhược
    điểm tồn tại trước đó trong công nghệ VPN truyền thống. Do vậy, trong đề tài
    khóa luận này em muốn giới thiệu công nghệ MPLS và dịch vụ MPLS VPN. Nội dung
    của đồ án sẽ tập trung trình bày những đặc điểm cơ bản của kiến trúc MPLS, tính
    ưu việt trong ứng dụng MPLS VPN, và các bước tiến hành cấu hình trên Router của
    hãng Cisco.

\textit{\textbf{Từ khóa:} MPLS, Chuyển mạch nhãn.}
\end{preamble}
\begin{contentlisting}
    \begin{contentlistingsection}{}
        \listoftables
    \end{contentlistingsection}

    \begin{contentlistingsection}{}
        \listoffigures
    \end{contentlistingsection}

    \begin{contentlistingsection}{Danh sách thuật ngữ}
        % Thêm các mục thuật ngữ dưới đây theo định dạng \item[Thuật ngữ] Mô tả
        \begin{description}
            \item[hehe] so hehe
        \end{description}
    \end{contentlistingsection}

    \begin{contentlistingsection}{Lời cam đoan}
        {\centering
        Tôi xin cam đoan rằng nội dung khóa luận này là kết quả lao động cá nhân của tôi,
        trừ những phần đã được trích dẫn và tham khảo rõ ràng. Tôi chịu trách nhiệm về
        những nội dung thuộc về mình và xin cam kết không vi phạm các quy định về đạo đức
        học thuật.\par}

        \vspace{2cm}
        \begin{center}
            \begin{minipage}[t]{0.3\textwidth}
                \centering
                \textbf{Chữ ký học sinh}
            \end{minipage}\hfill
            \begin{minipage}[t]{0.3\textwidth}
                \centering
                \textbf{Chữ ký học sinh}
            \end{minipage}\hfill
            \begin{minipage}[t]{0.3\textwidth}
                \centering
                \textbf{Chữ ký học sinh}
            \end{minipage}
        \end{center}
    \end{contentlistingsection}
\end{contentlisting}

% Bắt đầu nội dung chính (chương 1)
\chapter{Đặt vấn đề}
\label{ch:dat-van-de}

\noindent Đây là phần mở đầu chương 1 (Đặt vấn đề). Thêm nội dung chi tiết về bối cảnh,
vấn đề nghiên cứu, mục tiêu và phạm vi ở đây.

\end{document}
