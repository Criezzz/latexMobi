% !TeX root = ../main.tex
% Bảng dài tự ngắt trang: dùng ltablex (tabularx dạng longtable) và không đặt trong float
% In caption theo style của class `uetgraduation`
{\makeatletter
 \refstepcounter{a@table} % tăng số bảng
 \def\@captype{a@table} % đặt loại caption là bảng
 \def\@captiontext{\centering Bảng kiểm thử: Xóa bài viết} % nội dung caption
 \noindent\attachmentcaption\par\vspace{0.25cm} % in caption và giãn cách
\makeatother}

\setlength{\tabcolsep}{5pt}
\renewcommand{\arraystretch}{1.25}
\setlength{\arrayrulewidth}{1pt}

\begin{tabularx}{\textwidth}{|
  >{\centering\arraybackslash}p{1.2cm}|
  >{\raggedright\arraybackslash}X|
  >{\raggedright\arraybackslash}X|
  >{\raggedright\arraybackslash}X|
  >{\raggedright\arraybackslash}X|
  >{\centering\arraybackslash}p{2.2cm}|
}
\hline
\multicolumn{3}{|c|}{} &\multicolumn{3}{|c|}{\makecell[l]{Execute Date:\\ Tester:}}\\
\hline

\multirow{2}{*}{ID} & \multicolumn{1}{|c|}{\multirow{2}{*}{Description}} & \multicolumn{1}{|c|}{\multirow{2}{*}{Input}} & \multicolumn{2}{c|}{Output} & \multirow{2}{*}{Test status}\\
\cline{4-5}
 &  &  & Expected Output & Actual Output & \\
\hline

1 & Xóa bài viết hợp lệ & Tạo một bài viết thực hiện xóa bài viết bằng tùy chọn trên giao diện xem chi tiết bài viết & Có thông báo xác nhận xóa và bài viết được xóa không hiển thị trên diễn đàn sau khi xác nhận & Như mong đợi & Pass \\
\hline
2 & Hủy Xóa bài viết & Tạo một bài viết thực hiện xóa bài viết bằng tùy chọn trên giao diện xem chi tiết bài viết nhưng hủy xóa khi có thông báo xác nhận & Có thông báo xác nhận xóa, bài viết vẫn hiển thị trên diễn đàn sau khi nhấn hủy & Như mong đợi & Pass \\
\hline

\end{tabularx}
